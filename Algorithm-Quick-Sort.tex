\documentclass[main]{subfiles}
\begin{document}

%@@@@@@@@@@@@@@@@@@@@@@@@@@@@@@
% summarizes lecture 
% author:

\subsubsection{Quick Sort}
\renewcommand{\arraystretch}{1.5}
\definecolor{lgray}{gray}{0.95}
\definecolor{gray}{gray}{0.9}

\rowcolors{1}{lgray}{gray}

Conceptually, a quick sort works as follows:

\begin{enumerate}
\item Pick an element, called a pivot, from the array.
\item Reorder the array so that all elements with values less than the pivot come before the pivot, while all elements with values greater than the pivot come after it (equal values can go either way). After this partitioning, the pivot is in its final position. This is called the partition operation.
\item Recursively apply the above steps to the sub-array of elements with smaller values and separately to the sub-array of elements with greater values.
\end{enumerate}


\scriptsize
\begin{lstlisting}[language=Python]
import numpy as np
 
def incmatrix(genl1,genl2):
    m = len(genl1)
    n = len(genl2)
    M = None #to become the incidence matrix
    VT = np.zeros((n*m,1), int)  #dummy variable
 
    #compute the bitwise xor matrix
    M1 = bitxormatrix(genl1)
    M2 = np.triu(bitxormatrix(genl2),1) 
 
    for i in range(m-1):
        for j in range(i+1, m):
            [r,c] = np.where(M2 == M1[i,j])
            for k in range(len(r)):
                VT[(i)*n + r[k]] = 1;
                VT[(i)*n + c[k]] = 1;
                VT[(j)*n + r[k]] = 1;
                VT[(j)*n + c[k]] = 1;
 
                if M is None:
                    M = np.copy(VT)
                else:
                    M = np.concatenate((M, VT), 1)
 
                VT = np.zeros((n*m,1), int)
 
    return M
\end{lstlisting}


\end{document}