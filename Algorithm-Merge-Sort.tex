\documentclass[main]{subfiles}
\begin{document}

%@@@@@@@@@@@@@@@@@@@@@@@@@@@@@@
% summarizes lecture 
% author:

\subsubsection{Merge Sort}
\renewcommand{\arraystretch}{1.5}
\definecolor{lgray}{gray}{0.95}
\definecolor{gray}{gray}{0.9}

\rowcolors{1}{lgray}{gray}

Conceptually, a merge sort works as follows:

\begin{enumerate}
\item Divide the unsorted list into n sublists, each containing 1 element (a list of 1 element is considered sorted). (Time: O(n), Space: O(n) total)
\item Repeatedly merge sublists to produce new sorted sublists until there is only 1 sublist remaining. This will be the sorted list. (Time: O(n log(n), Space: O(n) total, O(n) auxiliary)
\end{enumerate}

\begin{tabular}{ll}
Worst case performance & O(n log n)\\
Best case performance & O(n log n) typical,O(n) natural variant\\
Average case performance & O(n log n)\\
Worst case space complexity & O(n) total, O(n) auxiliary\\
\end{tabular}

\scriptsize
\todo[inline]{Separate it properly into split and merge as in C++.}
\begin{lstlisting}[language=Java]
public int[] mergeSort(int array[])
// pre: array is full, all elements are valid integers (not null)
// post: array is sorted in ascending order (lowest to highest)
{
	// if the array has more than 1 element, we need to split it and merge the sorted halves
	if(array.length > 1)
	{
		// number of elements in sub-array 1
		// if odd, sub-array 1 has the smaller half of the elements
		// e.g. if 7 elements total, sub-array 1 will have 3, and sub-array 2 will have 4
		int elementsInA1 = array.length / 2;
		// we initialize the length of sub-array 2 to
		// equal the total length minus the length of sub-array 1
		int elementsInA2 = array.length - elementsInA1;
                // declare and initialize the two arrays once we've determined their sizes
		int arr1[] = new int[elementsInA1];
		int arr2[] = new int[elementsInA2];
		// copy the first part of 'array' into 'arr1', causing arr1 to become full
		for(int i = 0; i < elementsInA1; i++)
			arr1[i] = array[i];
		// copy the remaining elements of 'array' into 'arr2', causing arr2 to become full
		for(int i = elementsInA1; i < elementsInA1 + elementsInA2; i++)
			arr2[i - elementsInA1] = array[i];
		// recursively call mergeSort on each of the two sub-arrays that we've just created
		// note: when mergeSort returns, arr1 and arr2 will both be sorted!
		// it's not magic, the merging is done below, that's how mergesort works :)
		arr1 = mergeSort(arr1);
		arr2 = mergeSort(arr2);
		
		// the three variables below are indexes that we'll need for merging
		// [i] stores the index of the main array. it will be used to let us
		// know where to place the smallest element from the two sub-arrays.
		// [j] stores the index of which element from arr1 is currently being compared
		// [k] stores the index of which element from arr2 is currently being compared
		int i = 0, j = 0, k = 0;
		// the below loop will run until one of the sub-arrays becomes empty
		// in my implementation, it means until the index equals the length of the sub-array
		while(arr1.length != j && arr2.length != k)
		{
			// if the current element of arr1 is less than current element of arr2
			if(arr1[j] < arr2[k])
			{
				// copy the current element of arr1 into the final array
				array[i] = arr1[j];
				// increase the index of the final array to avoid replacing the element
				// which we've just added
				i++;
				// increase the index of arr1 to avoid comparing the element
				// which we've just added
				j++;
			}
			// if the current element of arr2 is less than current element of arr1
			else
			{
				// copy the current element of arr2 into the final array
				array[i] = arr2[k];
				// increase the index of the final array to avoid replacing the element
				// which we've just added
				i++;
				// increase the index of arr2 to avoid comparing the element
				// which we've just added
				k++;
			}
		}
		// at this point, one of the sub-arrays has been exhausted and there are no more
		// elements in it to compare. this means that all the elements in the remaining
		// array are the highest (and sorted), so it's safe to copy them all into the
		// final array.
		while(arr1.length != j)
		{
			array[i] = arr1[j];
			i++;
			j++;
		}
		while(arr2.length != k)
		{
			array[i] = arr2[k];
			i++;
			k++;
		}
	}
	// return the sorted array to the caller of the function
	return array;
}
\end{lstlisting}


\end{document}